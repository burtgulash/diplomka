\documentclass[11pt]{article}
\usepackage[T1]{fontenc}
\usepackage[utf8]{inputenc}
\usepackage[czech]{babel}
%\usepackage{a4wide}
\usepackage[numbers]{natbib}
\usepackage[pdfborder={0 0 0},colorlinks=true,linkcolor=green]{hyperref}
%\usepackage[bottom=2.5cm]{geometry}

\linespread{1.15}


\begin{document}
\section{Datové struktury pro vyhledávání v textu}
\subsection{Invertovaný index}
% nějaká historija atd.
Invertovaný index jednoduše řečeno vytvoří seznam dokumentů pro každé slovo,
pokud se v daném dokumentu slovo vyskytuje. Seznamu dokumentů pak říkáme
invertovaný seznam a všechny invertované seznamy jsou uloženy v paměti v tzv.
invertovaném souboru.

Slovem může být slovo lidské řeči, nebo jiné uspořádání po sobě jdoucích znaků.
Často se používají alternativní n-gramové invertované indexy, kde n-gramem
rozumíme posloupnost n po sobě jdoucích znaků.

Při použití klasických slov by se pro dokument s obsahem "máma mele maso"
vytvořily tři invertované seznamy pro každé vyskytující se slovo "máma", "mele"
a "maso".

\subsection{N-gramový invertovaný index}
Alternativně při použití trigramů (n-gramy, kde n = 3), by "slova" byla
\textbf{\_\_m, \_má, \_mám, áma, ma\_, a\_m, \_me, mel, ele, le\_, e\_m, \_ma,
mas, aso, so\_, o\_\_}. Ngramové indexy mají tu výhodu, že bez jakýchkoliv
dalších úprav s nimi lze vyhledávat přibližně. 

\url{https://swtch.com/~rsc/regexp/regexp4.html}




% word based vs ngram based
\subsection{Alternativní datové struktury}
Invertovaný index je doposud vítězem v rychlosti vyhledávání v textu, ale pro
některé speciální případy byly navrženy i jiné datové struktury, které dokáží
být kompaktnější než invertovaný index po kompresi, nebo umožňují vícero
seřazení při jednom uložení v paměti. Klasický invertovaný index má právě jedno
seřazení - nejčastěji podle vzrůstajícího id dokumentů nebo podle klesající
zvolené business metriky.

Waweletové stromy - dual sorted Index
Treap Index

Obě tyto alternativní datové struktury dosahují obrovských úspor díky své
kompresibilitě, čehož bychom využili, pokud bychom uvažovali rozsáhlá vstupní
data. Ale protože se zaměřuji na relativně malé objemy dat, avšak s vysokou
redundancí při prohledávání, nevyužiji tyto datové struktury. Dalším velkým
důvodem je, že jsem nenašel žádný výzkum, který by je aplikoval na prefixové,
natož přibližné vyhledávání. Na druhou stranu existuje více výzkumných týmů,
kteří se zabývají přibližným (fuzzy) prohledáváním na modifikovaném invertovaném
indexu.

\subsection{Hledání v invertovaném indexu}
Vstupním dotazem se rozumí posloupnost slov (dotazovaná slova) a speciálních
operátorů. Pro každé vstupní dotazované slovo se provede dotaz na první část
indexu - slovník. Ten vrátí všechny údaje nezbytné k tomu, abychom získali
odpovídající invertovaný seznam dokumentů. V případě prefixového a přibližného
dotazu slovník vrátí celou množinu odpovídajících slov.  V druhé fázi
vyhledávání se použijí nalezené invertované seznamy a provedou se s nimi
operace dle operátorů v dotazu uživatele. Třetí fází je seřazení výsledků podle
zvolené hodnotící funkce a sesbírání důkazů pro ty výsledky, které budou
zobrareny uživateli.

Invertované indexy podporují několik operací, se kterými může uživatel měnit
povahu svého dotazu. Nejčastěji jsou implementovány booleovské operátory AND,
OR, NOT a frázové vyhledávání - tedy omezení jen na ty výsledky, kde se slova
vyskytují těsně u sebe.

Apache Lucene operátory - fuzzy (nízká účinnost), změna ranku, ...

Pro naše požadavky, především kvůli jednoduchosti, bude postačovat operátor
AND, který navíc bude implicitní v dotazu. Tedy vstupním dotazem bude
posloupnost slov a výsledkem bude konjunkce výsledků nalezených pro každé z
nich. Funkcionalita disjunktivních dotazů (operátor OR) bude implicitní v tom,
že vyhledání slova ve slovníku vrátí disjunkci více přibližných slov nebo
prefixů v případě prefixového dotazu.  Operátoru NOT je občas potřeba, pokud
dojde k nejednoznačnostem ve výsledcích.  Uživatel může nejednoznačnost
identifikovat a zadat další dotaz, kde ji vyfiltruje právě pomocí operátoru
NOT. Otázkou je, zda-li by se mělo použítí operátoru NOT řídit stejnými
pravidly jako u běžného dotazovaného slova. Pokud by se použila množina slov v
blízkosti negovaného slova, mohlo by dojít k nežádoucímu přílišnému filtrování.

Poté, co v první části indexu - slovníku - nalezneme slova odpovídající
dotazovaným slovům a jejich invertované seznamy, přichází část, kdy se provedou
zadané operace s invertovanými seznamy.  Existují dvě hlavní možnosti - 1.
zpracovat nejprve každé slovo zvlášť a sloučit výsledek (Term At A Time - TAAT)
nebo 2. postupovat dokument po dokumentu napříč všemi slovy (Document At A Time
- DAAT).

\subsection{Prefixový a fuzzy invertovaný index}
Ačkoliv se jedná o dva různé požadavky, prefixové a fuzzy invertované indexy
mají ve výsledku podobné vlastnosti a tudíž jsou pro ně vhodné stejné datové
struktury. Pokud na chvíli necháme fuzzy požavky stranou, nastíním, jak lze
jednoduše rozšířit klasický invertovaný index o podporu prefixového
vyhledávání.

Invertovaný index má dvě hlavní části - slovník a tzv. invertovaný soubor.
Slovník obsahuje všechna slova nalezená ve zdrojovém textu a odkazy na
korespondující invertované seznamy v invertovaném souboru. Pro slovníky se
používají buď hashovací tabulky nebo je lze kompaktně uložit jako seřazený
seznam slov. V téhle struktuře se pak pomocí binárního vyhledávání najde
dotazované slovo a zaznamená se jeho pořadí v tomhle seznamu. Odkaz na
invertovaný seznam a další pomocná data pro tohle slovo se naleznou v jiném
seznamu tak, že zaznamenané pořadí se použije jako index do tohohle seznamu.

%\vspace{1cm}
%Příklad:
%
%\begin{center}
%\begin{tabular}{lr}
%\textbf{slovo}   & \textbf{ukazatel} \\
%\hline
%anderson          & 3  \\
%andrea            & 4  \\
%andulka           & 1  \\
%andy              & 2  \\
%kajak             & 13 \\ 
%kamna             & 14 \\
%karafiol          & 6  \\
%karate            & 12 \\ 
%karburátor        & 7  \\
%karel             & 11 \\
%karkulka          & 5  \\
%karma             & 8  \\
%karta             & 9  \\
%karty             & 10 \\ 
%\end{tabular}
%\end{center}

Pro ještě kompaktnější uložení se používá kompresní technika Front-Coding,
která využívá toho, že po sobě následující slova sdílí prefixy. Slovník se
rozdělí na bloky o konstantním počtu slov a prefix se pro tento blok uloží
zvlášť. Například při použití velikosti bloku 4 by slovník se slovy
\textbf{anderson, andrea, andulka, andy, kajak, kamna, karafiol, karate,
karburátor, karel, karkulka, karma, karta, karty} vypadal:
\textbf{\{and\}erson, rea, ulka, y | \{ka\}jak, mna, rafiol, rate | \{kar\}burátor, el, kulka, ma | \{kart\}a, y}

Reprezentace seřazeným seznamem slov je obzvlášť vhodná pro prefixové
vyhledávání, protože v takovém případě stačí nalézt první a poslední slovo ve
slovníku, které odpovídají dotazovanému prefixu. Všechna slova mezi těmito
dvěma odpovídají na prefixový dotaz.

Přibližné vyhledávání slov je s prefixovým podobné, protože většina podobných
slov bude v seřazeném slovníku blízko u sebe. Pokud vyhledáváme přibližná slova
do vzdálenosti 1, pak několik slov se bude lišit v prvním písmenu od
dotazovaného, ale většina rozdílů bude v pozdějších částech slova, a tedy ve
výsledku budou relativně blízko u sebe.

Pro rychlý průchod slovníkem s přibližným prohledáváním se hodí datová
struktura \textbf{kompaktní trie}. Je to vlastně seřazený seznam slov s tím, že
všechny společné prefixy pro všechny skupiny slov jsou uloženy jako rodičovské
uzly stromu n-árního stromu. Tohle rozložení nám umožní rychle dohledat prefix
a všechny jeho potomci ve stromu jsou odpovědí na prefixový dotaz. Přibližné
vyhledávání je efektivní, protože výpočet vzdálenostní funkce se opětovně
použije u všech slov, které sdílí prefix.

Kompaktní trie (ve skutečnosti jakákoliv trie) nám umožní nejen rychlé
vyhledání všech slov odpovídajících prefixu, ale zároveň i rychlé přibližné
vyhledání všech podobných slov podle zvolené distanční metriky.


\section{Odkazy}
\url{http://www.dcs.gla.ac.uk/~craigm/publications/lacour08efficiency.pdf}


% \section{Otevřené problémy}
% \subsection{Příliš krátký vstupní řetězec}



\end{document}
