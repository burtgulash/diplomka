\documentclass[11pt]{article}
\usepackage[T1]{fontenc}
\usepackage[utf8]{inputenc}
\usepackage[czech]{babel}
\usepackage{a4wide}
\usepackage[numbers]{natbib}
\usepackage[pdfborder={0 0 0}]{hyperref}
%\usepackage[bottom=2.5cm]{geometry}

\linespread{1.25}


\begin{document}
\subsection{Prefixový a fuzzy invertovaný index}
Ačkoliv se jedná o dva různé požadavky, prefixové a fuzzy invertované indexy
mají ve výsledku podobné vlastnosti a tudíž jsou pro ně vhodné stejné datové
struktury. Pokud na chvíli necháme fuzzy požavky stranou, nastíním, jak lze
jednoduše rozšířit klasický invertovaný index o podporu prefixového
vyhledávání.

Invertovaný index má dvě hlavní části - slovník a tzv. invertovaný soubor.
Slovník obsahuje všechna slova nalezená ve zdrojovém textu a odkazy na
korespondující invertované seznamy v invertovaném souboru. Pro slovníky se
používají buď hashovací tabulky nebo je lze kompaktně uložit jako seřazený
seznam slov. V téhle struktuře se pak pomocí binárního vyhledávání najde
dotazované slovo a zaznamená se jeho pořadí v tomhle seznamu. Odkaz na
invertovaný seznam a další pomocná data pro tohle slovo se naleznou v jiném
seznamu tak, že zaznamenané pořadí se použije jako index do tohohle seznamu.

%\vspace{1cm}
%Příklad:
%
%\begin{center}
%\begin{tabular}{lr}
%\textbf{slovo}   & \textbf{ukazatel} \\
%\hline
%anderson          & 3  \\
%andrea            & 4  \\
%andulka           & 1  \\
%andy              & 2  \\
%kajak             & 13 \\ 
%kamna             & 14 \\
%karafiol          & 6  \\
%karate            & 12 \\ 
%karburátor        & 7  \\
%karel             & 11 \\
%karkulka          & 5  \\
%karma             & 8  \\
%karta             & 9  \\
%karty             & 10 \\ 
%\end{tabular}
%\end{center}

Pro ještě kompaktnější uložení se používá kompresní technika Front-Coding,
která využívá toho, že po sobě následující slova sdílí prefixy. Slovník se
rozdělí na bloky o konstantním počtu slov a prefix se pro tento blok uloží
zvlášť. Například při použití velikosti bloku 4 by slovník se slovy
\textbf{anderson, andrea, andulka, andy, kajak, kamna, karafiol, karate,
karburátor, karel, karkulka, karma, karta, karty} vypadal:
\textbf{\{and\}erson, rea, ulka, y | \{ka\}jak, mna, rafiol, rate | \{kar\}burátor, el, kulka, ma | \{kart\}a, y}

Reprezentace seřazeným seznamem slov je obzvlášť vhodná pro prefixové
vyhledávání, protože v takovém případě stačí nalézt první a poslední slovo ve
slovníku, které odpovídají dotazovanému prefixu. Všechna slova mezi těmito
dvěma odpovídají na prefixový dotaz.

Přibližné vyhledávání slov je s prefixovým podobné, protože většina podobných
slov bude v seřazeném slovníku blízko u sebe. Pokud vyhledáváme přibližná slova
do vzdálenosti 1, pak několik slov se bude lišit v prvním písmenu od
dotazovaného, ale většina rozdílů bude v pozdějších částech slova, a tedy ve
výsledku budou relativně blízko u sebe.

Pro rychlý průchod slovníkem s přibližným prohledáváním se hodí datová
struktura \textbf{kompaktní trie}. Je to prakticky sežarezný seznam slov s tím,
že všechny společné prefixy pro všechny skupiny slov jsou uloženy jako
rodičovské uzly stromu. Tohle rozložení nám umožní rychle dohledat prefix a
všechny jeho potomci ve stromu jsou odpovědí na prefixový dotaz. Přibližné
vyhledávání je efektivní, protože výpočet vzdálenostní funkce se opětovně
použije u všech slov, které sdílí prefix.



\end{document}
